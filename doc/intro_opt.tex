\documentclass[12pt]{beamer}
\usepackage{../latex-sty/mypres}
\expandafter\def\expandafter\insertshorttitle\expandafter{%
  \insertshorttitle\hfill%
  \insertframenumber\,/\,\inserttotalframenumber}
\usefonttheme[onlymath]{serif}
\usepackage{listings}
\lstset{language=Python}

\begin{document}
\title[]{Введение в современную теорию \\ методов оптимизации}

\author[А. Катруца]{Aлександр Катруца \\[0.9cm] 
\includegraphics[scale=0.5]{logo_sirius}}

\date{Сочи 2017}
\begin{frame}
\maketitle
\end{frame}

\begin{frame}{Введение}

\begin{itemize}
\item Что такое методы оптимизации?
\item Какой математический аппарат используется?
\item Как развивалась теория методов оптимизации?
\item Какие главные результаты?
\end{itemize}

\end{frame}

\begin{frame}{Обозначения и типы задач}
\begin{equation*}
\begin{split}
& \min f(x)\\
\text{s.t. } & x \in X \subseteq \bbR^n
\end{split}
\end{equation*}

\begin{itemize}
\item Глобальный vs. локальный минимум
\item Условная vs.безусловная задача
\item Непрерывная vs. дискретная задача
\item Детерминированная vs. стохастическая задача
\end{itemize}
\end{frame}

\begin{frame}{Как сравнивать методы оптимизации?}
\begin{itemize}
\item Теоретическая сложность~--- об этом ниже
\item Масштабируемость
\item Время работы
\item Простота понимания и реализации
\end{itemize}
\end{frame}


\begin{frame}[fragile]{Общая схема}

\begin{lstlisting}[language=Python]
while (True):
    h = FindDirection(...)
    alpha = FindStepSize(...)
    x = x + alpha * h
    if StopCriterion(...):
    	break
\end{lstlisting}

\end{frame}

\begin{frame}{Методы первого порядка}
Идея: помимо значения функции в точке, использовать значение первой производной.

\begin{itemize}
\item градиентный спуск
\item метод сопряжённых градиентов
\item квазиньютоновские методы
\end{itemize}

\end{frame}

\begin{frame}{Методы второго порядка}

Идея: помимо значения функции в точке и значения первой производной использовать значения {\color{red}{второй}} производной в точке.

\begin{itemize}
\item Метод Ньютона
\end{itemize}

\end{frame}

\begin{frame}{Теория двойственности}

\end{frame}

\begin{frame}{Элементы теории сложности}

\end{frame}

\begin{frame}{Оптимальные методы}

\end{frame}

\begin{frame}{Заключение}

\end{frame}

\end{document}