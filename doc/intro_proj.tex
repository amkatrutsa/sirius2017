\documentclass[12pt]{beamer}
\usepackage{../latex-sty/mypres}
\expandafter\def\expandafter\insertshorttitle\expandafter{%
  \insertshorttitle\hfill%
  \insertframenumber\,/\,\inserttotalframenumber}
  
\begin{document}
\title[]{Введение в проект \\ <<Моделирование динамики дорожной ситуации в городе S>>}

\author[А. Катруца]{Научный руководитель: Ю. Е. Нестеров \\ Консультант: A. Катруца \\[0.5cm] 
\includegraphics[scale=0.5]{logo_sirius}}

\date{Сочи 2017}


\begin{frame}
\maketitle
\end{frame}

\begin{frame}{Цели проекта}
\begin{itemize}
\item Знакомство с современным состоянием теории транспортного моделирования 
\item Знакомство с современными методами оптимизации
\item Получение опыта реализации численных методов оптимизации и исследования их свойств
\item Получение результатов моделирования дорожной ситуации в реальном городе
\end{itemize}
\end{frame}

\begin{frame}{Этапы проекта}
\begin{enumerate}
\item Сбор данных дорожных сетей из открытых источников
\item Реализация базовых методов решения задачи поиска равновесия
\item Сравнение методов и анализ результатов
\item Презентация результатов
\end{enumerate}
\end{frame}

\begin{frame}{Сбор данных}
\begin{enumerate}
\item Поиск размеченных дорожных сетей в открытых источниках
\item Разметка данных вручную на основании картографических сервисов
\end{enumerate}
\end{frame}

\begin{frame}{Реализация базовых методов решения}
\begin{itemize}
\item Метод зеркального спуска
\item Универсальный метод подобных треугольников
\end{itemize}

Об этих методов будет рассказано позднее.
\end{frame}

\begin{frame}{Сравнение методов}
\begin{itemize}
\item Время работы
\item Скорость сходимости
\item Масштабируемость
\end{itemize}
\end{frame}

\begin{frame}{Презентация результатов}
\begin{itemize}
\item Финальный отчёт о работе над проектом
\item Заполняется по мере выполнения текущих задач
\item Содержит следующие элементы:
\begin{itemize}
\item обоснование актуальности проекта
\item формальная постановка задачи
\item анализ задачи и мотивация её введения
\item формальное описание методов решения
\item теоретическое сравнение методов решения
\item практическое сравнение методов решения
\item вывод о применимости рассматриваемых методов для решения поставленной задачи
\end{itemize}
\item Технически желательно делать в \LaTeX
\end{itemize}
\end{frame}

\begin{frame}{Математическая часть}
\begin{enumerate}
\item Линейная алгебра
\item Методы оптимизации
\item Элементы теории графов
\end{enumerate}
\end{frame}

\begin{frame}{Заключение}
\begin{itemize}
\item Проект посвящён моделированию транспортных систем
\item Цели проекта возможно скорректировать по мере выполнения текущих задач
\item Результат~--- демо-скрипт для сравнения базовых методов оптимизации при решении задачи поиска равновесия в транспортной сети
\item Требуются базовые навыки программирования и интерес к теме~--- остальное будет рассказано по мере выполнения проекта
\end{itemize}
\end{frame}

\end{document}